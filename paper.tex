\documentclass[11pt]{article}

% Packages
\usepackage{graphicx}
\usepackage{authblk}
\usepackage{amsmath}
\usepackage[a4paper,margin=2.5cm]{geometry}
\usepackage{hyperref}
\usepackage{caption}
\usepackage{float}
\usepackage[numbers]{natbib}

% Title and authors
\title{A Comprehensive Dataset for Vessel Detection and Localization Using Distributed Acoustic Sensing in Submarine Cables}

\author[1]{Erick Eduardo Ramirez-Torres}
\author[1]{Javier Macias-Guarasa}
\author[1]{Daniel Pizarro-Perez}
\author[2]{Javier Tejedor}
\author[1]{Sira Elena Palazuelos-Cagigas}
\author[1]{Pedro J. Vidal-Moreno}
\author[1]{Sonia Martin-Lopez}
\author[1]{Miguel Gonzalez-Herraez}
\author[3]{Roel Vanthillo}

\affil[1]{Departamento de Electr\'onica, Universidad de Alcal\'a, Alcal\'a de Henares, Spain}
\affil[2]{Universidad San Pablo-CEU, Boadilla del Monte, Spain}
\affil[3]{Marlinks, Leuven, Belgium}

\date{}

\begin{document}

\maketitle

\begin{abstract}
We present a curated dataset designed for the development and evaluation of machine learning models for vessel detection and localization using Distributed Acoustic Sensing (DAS) in submarine optical fiber cables. The dataset comprises preprocessed DAS signal features, synchronized with anonymized vessel proximity indicators derived from AIS data, acquired over a ten-day continuous period. While the original AIS data and precise fiber location are restricted due to confidentiality agreements, the released dataset includes features and labels supporting reproducible experiments under realistic marine conditions.
\end{abstract}

\section*{Background \& Summary}
Submarine cables are vital for global communication and power transmission but remain vulnerable to accidental damage and sabotage. Recent technological advances have enabled the use of Distributed Acoustic Sensing (DAS) systems to repurpose these cables into dense acoustic sensor arrays. DAS provides continuous monitoring capabilities independent of cooperative systems like AIS, and offers robust performance in challenging environmental conditions.

Despite growing interest, publicly available datasets supporting machine learning development in this context are lacking. This work addresses this gap by releasing a large-scale, preprocessed dataset based on DAS recordings over a buried submarine cable. Although proprietary constraints prevent us from sharing raw AIS data or the exact cable route, the dataset includes carefully anonymized and aggregated features that retain the essential characteristics needed for vessel detection and proximity estimation tasks.

\section*{Methods}
The dataset was built from ten consecutive days (June 16–25, 2023) of continuous DAS signal acquisition over a 26 km buried submarine cable. A commercial Alcatel OptoDAS interrogator sampled strain data at 3125 Hz, using optical pulse compression reflectometry and a 10 m gauge length. The spatial resolution was 10.21 m, producing 2600 channels across the cable.

To curate the dataset:
\begin{itemize}
  \item Raw differential-phase data was processed using phase unwrapping, spike correction, integration, and normalization to produce strain estimates.
  \item Features were computed as energy in 100 logarithmically spaced frequency bands between 4 Hz and 98 Hz, excluding the 49–51 Hz interval to suppress 50/100 Hz system noise.
  \item AIS data, reporting vessel position, heading, and type at 1–3 minute intervals, was interpolated to 1 Hz and aligned temporally with DAS data.
  \item Labels were derived from the interpolated distance to the closest vessel, producing:
    \begin{itemize}
      \item Binary labels: vessel within 1000 m (Class 0) vs. beyond (Class 1).
      \item Continuous distance labels: closest vessel distance at each timestamp.
    \end{itemize}
\end{itemize}

The processing pipeline ensures synchronized, anonymized, and reproducible training data without disclosing raw DAS signals or geolocation.

\begin{figure}[H]
  \centering
  \includegraphics[width=0.9\textwidth]{pipeline_diagram_placeholder.png}
  \caption{Dataset generation pipeline: from DAS raw signals and AIS metadata to labeled, preprocessed spectral features.}
\end{figure}

\section*{Data Records}
The dataset is stored in a structured HDF5 format, organized as follows:
\begin{itemize}
  \item \texttt{/features}: Float32 arrays of shape (N, 100), spectral features per frame.
  \item \texttt{/labels/distance}: Float32 array (N,), estimated distance to the nearest vessel.
  \item \texttt{/labels/class}: Int array (N,), binary detection label.
  \item \texttt{/meta/timestamps}: Datetime64 array (N,), timestamps aligned to UTC.
  \item \texttt{/meta/sensor_ids}: Int array (N,), identifies central channel for each spatial window.
\end{itemize}

\begin{table}[H]
  \centering
  \caption{Summary of released dataset contents}
  \begin{tabular}{@{}lll@{}} \toprule
  Data Component & Description & Dimensions \\ \midrule
  Spectral Features & Log-spaced band energy (4–98 Hz) & 74771 $\times$ 100 \\
  Binary Labels & Vessel $\lt$ 1000 m (1/0) & 74771 \\
  Distance Labels & Closest vessel distance [m] & 74771 \\
  Temporal Info & UTC-aligned timestamps & 74771 \\
  Sensor Info & Spatially averaged from 250 sensors & 74771 \\
  \bottomrule
  \end{tabular}
\end{table}

\section*{Technical Validation}
We validated the dataset by training two ML models:
\begin{itemize}
  \item XGBoost classifier: F1-score $>$ 90\% on vessel detection at 1000 m.
  \item XGBoost regressor: Mean Absolute Error = 141 m (rMAE = 14.1\%).
\end{itemize}

Models were evaluated using 10-fold cross-validation, assigning each day to a different fold. This avoids temporal leakage and maximizes vessel diversity across training and testing.

We also analyzed spectral features to verify discriminative structure (2–100 Hz range) and inspected time-aligned energy maps vs. AIS-derived distance profiles, confirming strong signal correlation with vessel proximity.

\begin{figure}[H]
  \centering
  \includegraphics[width=0.85\textwidth]{placeholder_features.png}
  \caption{Feature map: energy evolution across fiber channels compared to vessel proximity (bottom).}
\end{figure}

\section*{Usage Notes}
The dataset supports binary classification and regression tasks. Recommended usage includes:
\begin{itemize}
  \item Evaluation of signal processing and ML models for DAS-based vessel detection.
  \item Analysis of spatial and temporal resolution tradeoffs in DAS sensing.
  \item Testing domain adaptation techniques across maritime environments.
\end{itemize}
Please note that the dataset does not include raw DAS or AIS data, nor geospatial coordinates. All experiments must be based on the released feature-label pairs.

\section*{Code Availability}
All code for feature extraction, preprocessing, model training, and evaluation is available at \url{https://github.com/UAH-PSI/das-vessel-detection}.

%\section*{References}
\begin{thebibliography}{9}
\bibitem{landro2022} Landr\o, M. et al. Sensing whales, storms, ships and earthquakes using an Arctic fibre optic cable. \textit{Sci. Rep.} \textbf{12}, 1–12 (2022).
\bibitem{rivet2021} Rivet, D. et al. Preliminary assessment of ship detection using DAS. \textit{JASA} \textbf{149}, 2615–2627 (2021).
\bibitem{tejedor2019} Tejedor, J. et al. A contextual GMM-HMM smart fiber optic surveillance system. \textit{JLT} \textbf{37}, 4514–4522 (2019).
\end{thebibliography}

\section*{Author Contributions}
E.E.R.-T., J.M.-G., and D.P.-P. designed the study. J.T. and R.V. coordinated data access and metadata generation. P.J.V.-M. and S.E.P.-C. implemented the processing and modeling framework. S.M.-L. and M.G.-H. supervised optical sensing aspects. All authors contributed to writing and reviewing the manuscript.

\section*{Competing Interests}
The authors declare no competing financial or non-financial interests.

\section*{Acknowledgements}
This work was supported by the Spanish Ministry of Science and Innovation and the European Union through grants PSI (PLEC2021-007875), REMO (CPP2021-008869), and SUBMERSE (101095055), among others. We acknowledge the cable operator for providing access to DAS and AIS data under confidentiality agreements, and Artemisa for computational resources.

\end{document}
